\documentclass[parskip=half]{scrartcl}

\usepackage[hidelinks]{hyperref}

\usepackage{tgheros}

\usepackage{microtype}
\usepackage{fontspec}

\usepackage[
    backend=biber,
    style=ieee,
]{biblatex}
\addbibresource{ref.bib}

\setmainfont{Georgia}
\setsansfont{Helvetica Neue}

\KOMAoptions{DIV=calc}

\begin{document}

\input{../author}

\begin{center}
    \Large
    \textsf{\textbf{Vastaamo  Mental Health Platform}}
        
    \vspace{0.4cm}
    \large
    Homework 1: Privacy Preserving Technologies
        
    \vspace{0.4cm}
    \docauthor{}
       
    \vspace{0.9cm}
\end{center}

\subsubsection*{Description of the system}

% Description of the system - what does/would it do? (Up to 1 paragraph)

Vastaamo provided private online mental-health services to patients. Patients
were offered online counselling, but in-person therapy sessions were also
available. Therapists would write and upload therapy notes of patient sessions
in the company's browser-based note system. Scheduling and invoicing would also
happen through the online platform \cite{wired1}.

\subsubsection*{Stakeholders}

% Stakeholders who would be touching the data in some way. Please distinguish
% data owners, the service provider and result users. For each stakeholder,
% state their role in the system. (Up to 3 short paragraphs)

\begin{itemize}
    \item The patient: patients are the data subject. It is data about them that
    is stored by Vastaamo. They give this data during therapy sessions.
    \item The therapist: therapists are the data owners. They input the personal
    data (i.e., the therapy notes) into the system, and it is in their interest
    that it is correct. Therapists are also the result users, since they use the
    data that they input.
    \item Vastaamo: Vastaamo is the service provider. They provide the web
    interface and databases. In a sense, Vastaamo is a joint data owner with the
    therapists.
\end{itemize}

It is unclear whether the patient can directly manage details such as their
contact details, or if these are also input into the system by the therapist.
While the patients do benefit (hopefully) from Vastaamo, they do not get to
access the therapy notes.

\subsubsection*{Explicit properties}

% Explicit properties. Describe the communicated functionality of the system,
% expected impact, used technology, operating concepts and if any privacy
% control measures have been used. (Up to 3-4 paragraphs of reasonable length)

Vastaamo designed its bespoke system for patient data management. Therapists
would use a web-interface to write their notes during patient sessions. This
would allow them to have all notes for all patients in a single place, for easy
search and review \cite{wired1}.

While not all technical details are explicit and known, data suggests that
records were stored on a MySQL system without anonymisation or encryption
\cite{wired1}. The database contained the contact details, national
identification numbers and the therapy notes of patients \cite{berggren}. Not
only was the note-taking portal remotely accessible, but the database servers
themselves were also open to the world. Moreover, the database root account was
not password protected \cite{ombudsman}.

The system was not connected to the national health data repository ``Kanta'',
so security and privacy practices where laxly regulated. Vastaamo only had to
self-certify that their system met government regulations \cite{wired1}.

\textit{Note: I assume that when sources mention a lack of anonymisation, they
mean also a lack of pseudonymisation, as it would sound impractical for
therapists to recall and search through anonymised patient notes.}

\subsubsection*{Emergent properties}

% Emergent/implicit properties. Analyse the system from the point of impact to
% the personal/confidential data. What kind of leakages are possible? How could
% these leakages be used? What could the impact to data owners/subjects be? You
% may refer to sources you've read here, but my expectations is that you are
% capable of providing additional analysis of your own (Up to 3-4 paragraphs of
% reasonable length)

Because of the highly sensitive nature of the personal data collected by
Vastaamo, any leakage could have (and had) disastrous consequences for
the patients involved. The ``least damaging'' leak would concern only
administrative patient records, such as their contact information and national
identification numbers. For leaked therapy notes, the impact is far worse.

In case of a non-public leak, this data could be used for patient extortion,
(e.g., money, favours). As data ended up being leaked publicly, patients became
vulnerable to discrimination from both regular citizens but for example also by
employers. Even more, if leaked records contain illegal practices of a patient,
e.g., substance abuse, it could leave patients open for criminal investigations.
It is easy to see how such a leak could destroy lives of people affected.

It is worth noting, that even if records were leaked without associated patient
data (e.g., name), then due to the deeply personal nature of the data, someone
could potentially process the leaked data and tie it back to a person they know.
For example, it is reasonable to assume that some patients would mention the
names of their friends and family, and the name of their school or workplace.

As therapy is based on patient-therapist trust, and this trust should extend to
the service provider, then a leak of all patient records is a business-ending
event. Not only will clients no longer turn towards the business, but the legal
fees involved with such a data breach can be astronomical. On top of causing
bankruptcy for the company, management and/or employees could face severe
charges against their person, such as for malpractice or gross negligence.

% References: source material that you have used, preferably public materials I
% could review and compare to your work (see above for the use of confidential
% source materials).

\printbibliography

\end{document}