\documentclass[parskip=half]{scrartcl}

\usepackage[hidelinks]{hyperref}

\renewcommand*{\thefootnote}{(\arabic{footnote})}

\usepackage{microtype}
\usepackage{fontspec}
\usepackage{unicode-math}

\setmainfont{Georgia}
\setsansfont{Helvetica Neue}
\setmathfont{Stix Two Math}

\KOMAoptions{DIV=calc}

\begin{document}

\input{../author}

\begin{center}
    \Large
    \textsf{\textbf{Differential Privacy}}
        
    \vspace{0.4cm}
    \large
    Homework 5: Privacy Preserving Technologies
        
    \vspace{0.4cm}
    \docauthor{}
       
    \vspace{0.9cm}
\end{center}

\tableofcontents

\section{Differential privacy of query Q1}

\subsection{Choosing the \texorpdfstring{$\epsilon$}{ε}}

\begin{enumerate}
    \item \textit{What is the global sensitivity of the query Q1?}
    
    For Q1, all votes for all candidates are counted, and the total of votes
    for each candidate is returned. As such, any vote by a voter will increment
    the total of votes for one candidate by one (assuming a single-vote
    election), regardless of who they vote for.

    We can thus say that for Q1, the difference of vote counts per candidate
    for two adjacent tables is $1$, and hence the global sensitivity is
    $\Delta q_1 = 1$.
    
    \item \textit{Which $\epsilon$ should be taken so that the probability of
    getting the correct result is at least $0.9$?}

    We wish that the probability of getting the correct election outcome be at
    least $0.9$. This is only the case where all three vote totals are correct.
    As such, each vote total must be correct with probability at least
    $\sqrt[3]{0.9}$, assuming that the noise distribution is the same for all
    three candidates. Conversely, this means that we do not get the correct
    election outcome should we get a wrong vote total for any candidate, and
    the probability of this happening, i.e., the probability of failure, must
    be smaller than $\beta = 1-\sqrt[3]{0.9}$.

    A difference of even one vote from the true vote count of a candidate can
    change the outcome of the election. Because we round noise-added values to
    the nearest integer value, and values themselves are already integers, we
    require noise to be below $0.5$ to ensure the ``noised'' value is
    rounded down. This avoids adding one vote to the summed candidate vote
    count, and is essentially equivalent of applying no noise.
    
    More specifically, in the given case, rounding the noise to the nearest
    integer before adding it to the vote count is equivalent to rounding the
    result, and hence does not go against the post-processing theorem. We can
    hence say that $|\textit{noise}| < |\alpha| = 0.5$ must hold with
    probability of at least $\beta$\footnotemark{}, where $\alpha$ is the
    accuracy.

    \footnotetext{$\beta$ is an exclusive upper bound for the error, as
    $1-\sqrt[3]{0.9}$ is included in the probability of success.}

    We observe that since Q1 releases information about all three candidates,
    but a voter can only vote for one candidate, the total vote counts for each
    candidate are disjoint subsets with regards to the vote of a single voter.
    We then decompose the query Q1 into three sub-queries consisting of the
    vote count for each candidate. We hence have $Q1_f, Q1_w, Q1_b$, with
    $Q1_f \cup Q1_w \cup Q1_b = Q1$, and $Q1_x \cap Q1_y = \varnothing$ for
    $x, y\in, \{f, w, b\}$, and $x\neq y$.

    If we consider the noise distribution to be the same for all three
    candidates, we have that $Q1_f, Q1_w$ and $Q1_b$ provide equivalent
    $\epsilon$-differential privacy, which is also the total privacy level of
    Q1 due to parallel composition.

    We have
    $$
        \alpha = \frac{\Delta q_1}{\epsilon}\ln{\frac{1}{\beta}}
        \quad\Leftrightarrow\quad
        -\frac{\alpha\epsilon}{\Delta q_1} = \ln{\beta}
        \quad\Leftrightarrow\quad
        \beta = e^{-\frac{\alpha\epsilon}{\Delta q_1}}.
    $$
    
    Thus, if we want $\beta < 1 - \sqrt[3]{0.9}$ with $\Delta q_1 = 1$, we have
    \begin{alignat*}{2}
        \quad &&
            1 - \sqrt[3]{0.9} &> e^{-\frac{\alpha\epsilon}{\Delta q_1}}\\
        \Leftrightarrow\quad &&
            \ln{\left(1 - \sqrt[3]{0.9}\right)} &> -\alpha\epsilon\\
        \Leftrightarrow\quad &&
            \epsilon &> -2 \ln{\left(1 - \sqrt[3]{0.9}\right)} \approx 6.733.
    \end{alignat*}

    To get the correct result with probability at least $0.9$, we must take
    $\epsilon$ of at least $6.733$\footnote{$6.733$ is already large enough
    because we rounded up.}.
\end{enumerate}

\subsection{Dealing with negative counts}

\textit{Which of the following modifications would keep the system
$\epsilon$-differentially private with the same value of $\epsilon$ due to the
post-processing theorem? Why?}

\begin{enumerate}
    \item \textit{Sample only positive noise, i.e. from the interval
    $[0, \infty)$.}

    According to the post-processing theorem, processing must not depend on the
    noise and/or private data added by the mechanism. Thus, this case does not
    fall under the post-processing theorem, and the modified system may no
    longer be $\epsilon$-differentially private, as it may leak information
    about the underlying data. More specifically, we have altered the noise
    distribution and introduced value bias.

    \item \textit{If the true vote count of some candidate is some value $y$,
    sample the noise from the interval $[-y, \infty)$.}

    The vote count of some candidate is private data, i.e., data we seek to
    protect. The post-processing theorem states that processing must not depend
    on private data. Hence, this system may no longer be
    $\epsilon$-differentially private, as it may leak information about the
    underlying data. More specifically, we have altered the noise distribution
    and introduced value bias.

    \item \textit{If the noisy vote count of some candidate is negative, then
    re-sample the noise until the result becomes positive.}
    
    The approach may reveal timing information exploitable by side-channel
    attacks for an active adversary. We have also altered the noise
    distribution, as our result is equivalent to once more applying noise only
    from the $[-y, \infty)$ interval. The system is therefore
    not $\epsilon$-differentially private.

    We note that we understand this modification to re-add the noise to the
    original value. The situation would be different if noise was added to
    already noised output.

    \item \textit{If the noisy vote count of some candidate is negative, then
    round this negative count up to $0$.}

    This modification keeps the system $\epsilon$-differentially private. This
    processing is deterministic, and can be done with anyone having access to
    the $\epsilon$-differentially private output of the initial mechanism.
\end{enumerate}

\section{Differential privacy for query Q2}

\subsection{Choosing the \texorpdfstring{$\epsilon$}{ε}}

\begin{enumerate}
    \item Which $\epsilon$ should be taken so that the probability of getting a
    correct result should be at least $0.9$?


\end{enumerate}

\end{document}
