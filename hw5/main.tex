\documentclass[parskip=half]{scrartcl}

\usepackage[hidelinks]{hyperref}

\renewcommand*{\thefootnote}{(\arabic{footnote})}

\usepackage{microtype}
\usepackage{fontspec}
\usepackage{unicode-math}

\setmainfont{Georgia}
\setsansfont{Helvetica Neue}
\setmathfont{Stix Two Math}

\KOMAoptions{DIV=calc}

\begin{document}

\input{../author}

\begin{center}
    \Large
    \textsf{\textbf{Differential Privacy}}
        
    \vspace{0.4cm}
    \large
    Homework 5: Privacy Preserving Technologies
        
    \vspace{0.4cm}
    \docauthor{}
       
    \vspace{0.9cm}
\end{center}

\tableofcontents

\section{Differential privacy of query Q1}

\subsection{Choosing the \texorpdfstring{$\epsilon$}{ε}}

\begin{enumerate}
    \item \textit{What is the global sensitivity of the query Q1?}
    
    For Q1, all votes for all candidates are counted, and the total of votes
    for each candidate is returned. As such, any vote by a voter will increment
    the total of votes for one candidate by one (assuming a single-vote
    election), regardless of who they vote for.

    We can thus say that for Q1, the difference of vote counts per candidate
    for two adjacent tables is $1$, and hence the global sensitivity is
    $\Delta q_1 = 1$.
    
    \item \textit{Which $\epsilon$ should be taken so that the probability of
    getting the correct result is at least $0.9$?}

    We wish that the probability of getting the correct election outcome be at
    least $0.9$. This is only the case where all three vote totals are correct.
    As such, each vote total must be correct with probability at least
    $\sqrt[3]{0.9}$, assuming that the noise distribution is the same for all
    three candidates. Conversely, this means that we do not get the correct
    election outcome should we get a wrong vote total for any candidate, and
    the probability of this happening, i.e., the probability of failure, must
    be smaller than $\beta = 1-\sqrt[3]{0.9}$.

    A difference of even one vote from the true vote count of a candidate can
    change the outcome of the election. As such, we require that the noise
    added to a result is less than one. Because we round noise to the nearest
    integer value, we require noise to be below the accuracy to ensure it is
    rounded down, i.e., $|\textit{noise}| < |\alpha| = 0.5$. This must hold
    with probability of at least $\beta$\footnote{$\beta$ is an exclusive
    upper bound for the error, as $1-\sqrt[3]{0.9}$ is included in the
    probability of success.}.

    We observe that since Q1 releases information about all three candidates,
    but a voter can only vote for one candidate, the total vote counts for each
    candidate are disjoint subsets with regards to the vote of a single voter.
    We then decompose the query Q1 into three sub-queries consisting of the
    vote count for each candidate. We hence have $Q1_f, Q1_w, Q1_b$, with
    $Q1_f \cup Q1_w \cup Q1_b = Q1$, and $Q1_x \cap Q1_y = \varnothing$ for
    $x, y\in, \{f, w, b\}$, and $x\neq y$.

    If we consider the noise distribution to be the same for all three
    candidates, we have that $Q1_f, Q1_w$ and $Q1_b$ provide equivalent
    $\epsilon$-differential privacy, which is also the total privacy level of
    Q1.

    We have
    $$
        \alpha = \frac{\Delta q_1}{\epsilon}\ln{\frac{1}{\beta}}
        \quad\Leftrightarrow\quad
        -\frac{\alpha\epsilon}{\Delta q_1} = \ln{\beta}
        \quad\Leftrightarrow\quad
        \beta = e^{-\frac{\alpha\epsilon}{\Delta q_1}}.
    $$
    
    Thus, if we want $\beta < 1 - \sqrt[3]{0.9}$ with $\Delta q_1 = 1$, we have
    \begin{alignat*}{2}
        \quad &&
            1 - \sqrt[3]{0.9} &> e^{-\frac{\alpha\epsilon}{\Delta q_1}}\\
        \Leftrightarrow\quad &&
            \ln{\left(1 - \sqrt[3]{0.9}\right)} &> -\alpha\epsilon\\
        \Leftrightarrow\quad &&
            \epsilon &> -2 \ln{\left(1 - \sqrt[3]{0.9}\right)} \approx 6.733.
    \end{alignat*}

    To get the correct result with probability at least $0.9$, we must take
    $\epsilon$ of at least $6.733$\footnote{$6.733$ is already large enough
    because we rounded up.}.
\end{enumerate}

\subsection{Dealing with negative counts}

\textit{Which of the following modifications would keep the system
$\epsilon$-differentially private with the same value of $\epsilon$ due to the
post-processing theorem? Why?}

\begin{enumerate}
    \item \textit{Sample only positive noise, i.e. from the interval
    $[0, \infty)$.}

    According to the post-processing theorem, processing is done on a
    mechanism's output, and not the on the noise being sampled. That is, only
    the data with the noise already added is processed. Hence this case does
    not fall under the post-processing theorem, and we cannot assess this
    modified system's $\epsilon$-differential privacy under the theorem.

    \item \textit{If the true vote count of some candidate is some value $y$,
    sample the noise from the interval $[-y, \infty)$.}

    According to the post-processing theorem, processing is done on a
    mechanism's output, and not the on the noise being sampled. That is, only
    the data with the noise already added is processed. Hence this case does
    not fall under the post-processing theorem, and we cannot assess this
    modified system's $\epsilon$-differential privacy under the theorem.

    \item \textit{If the noisy vote count of some candidate is negative, then
    re-sample the noise until the result becomes positive.}
    \item \textit{If the noisy vote count of some candidate is negative, then
    round this negative count up to $0$.}
\end{enumerate}

\section{Differential privacy for query Q2}

\subsection{Choosing the \texorpdfstring{$\epsilon$}{ε}}

\begin{enumerate}
    \item Which $\epsilon$ should be taken so that the probability of getting a
    correct result should be at least $0.9$?


\end{enumerate}

\end{document}
